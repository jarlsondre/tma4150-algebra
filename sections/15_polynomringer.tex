\section{Polynomringer}
\begin{definition}{}{}
	La $R$ være en ring. Da definerer vi følgende:
	\begin{enumerate}
		\item Et \textbf{polynom} med koeffisienter i $R$ er definert som
		      \begin{align}
			      f(x) = a_0 + a_1x + \dots + a_nx^n
		      \end{align}
		      hvor $a_i \in R$. Vi sier at $f(x)$ har \textbf{grad} $n$ dersom $n$ er den største indeksen
		      slik at $a_n \neq 0$ og skriver $\deg f(x) = n$.
		\item Vi definerer \textbf{polynomringen} over $R$ som
		      \begin{align}
			      R\sb{x} = \cb{p(x)\mid p\text{ er et polynom med koeffisienter i }R}
		      \end{align}
		      En slik polynomring har følgende ringstruktur:

		      Dersom $p(x) = a_0 + a_1x + \dots + a_nx^n$ og $q(x) = b_0 + b_1x + \dots + b_nx^n$
		      så vil
		      \begin{align}
			      p(x) + q(x) & = (a_0 + b_0) + (a_1 + b_1)x + \dots + (a_n + b_n)x^n \\
			      p(x)q(x)    & = a_0b_0 + (a_0b_1 + a_1b_0)x + \dots,
		      \end{align}
		      altså slik som vi er vandte med fra før.
	\end{enumerate}
\end{definition}
\textbf{Merk}
\begin{enumerate}
	\item $R[x]$ kommutativ $\iff R$ kommutativ.
	\item $\deg (p(x) + q(x)) \leq \max(\deg p(x), \deg q(x))$ og
	      $\deg (p(x)q(x)) \leq \deg p(x) + \deg q(x)$.

	      For eksempel: Dersom vi er i $\mathbb{Z}_8[x]$,
	      så har vi $(4x^2 0 3)(2x+1) = 8x^3 + 4x^2 + 6x + 3 = 4x^2 + 6x + 3$ siden $8x^3$ forsvinner
	      i $\mathbb{Z}_8$.
	\item $R[x]$ integritetsområde $\iff R$ integritetsområde

	      I så fall er $\deg (p(x)q(x)) = \deg p(x) + \deg q(x)$.

        Spesielt så har vi at dersom $F$ er en kropp så er $F[x]$ et integritetsområde (men ikke 
        en kropp).
      \item Den multiplikative identiteten i $R[x]$ er $p(x) = 1$. 
\end{enumerate}

\begin{theorem*}{22.4}{}
  La $F\subseteq E$ være kropper og $\alpha\in E$. Da er  
  $\phi_\alpha : F[x] \ra E, p(x)\mapsto p(\alpha)$ en ringhomomorfi. Denne kaller vi 
  \textbf{evaluering} i $\alpha$.  
\end{theorem*}

\textbf{Eksempel}: Se på $\mathbb{Q}\subseteq \mathbb{R}$ og $\alpha=\sqrt{2}\in \mathbb{R}$.
Definer $\phi_{\sqrt{2}} : \ra \mathbb{R}, p(x) \mapsto p(\sqrt{2})$. Da ser vi at for 
$p(x)=x^2-2$ så er $\phi_{\sqrt{2}}(p)=0$ og for $q(x)=x^3+1$ så er 
$\phi_{\sqrt{2}}(q)=2\sqrt{2}+1$.

\begin{definition}{Rot}{}
  La $F\subseteq E$ være kropper, $p(x)\in F[x]$ og $\alpha\in E$. Da sier vi at $\alpha$ er en
  \textbf{rot} i $p(x)$ dersom $p(\alpha)=0$.
\end{definition}

\textbf{Eksempler}:
\begin{enumerate}
  \item Polynomet $p(x)=x^2+1$ i $\mathbb{R}$ har ingen røtter.
  \item Polynomet $p(x)=x^2+x+1\in \mathbb{Z}_7[x]$ har to røtter i $\mathbb{Z}_7$: $\cb{2,4}$.
\end{enumerate}
