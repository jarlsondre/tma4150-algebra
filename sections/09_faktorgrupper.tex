\section{Faktorgrupper}

\textbf{Mål}: For en gruppe $G$ og visse undergrupper $H\leq G$ så vil vi lage en ny gruppe
$G/H$ hvor elementene i $G/H$ er restklassene til $H$ i $G$.

\begin{definition}{Normal Undergruppe}{}
	Vi kaller en undergruppe $H\leq G$ \textbf{normal} dersom $gH=Hg\ \forall g\in G$.
\end{definition}

\textbf{Eksempler}:
\begin{enumerate}
	\item $H=\cb{e}\implies gH=\cb{g}=Hg$ for alle $g\in G$, så her er $H$ normal.
	\item Dersom $G$ er abelsk så er $H\leq G$ automatisk normal.
	\item La $G=S_3$ tolket som symmetrier på et triangel,
	      $\cb{\rho_0, \rho_1, \rho_2, \mu_1, \mu_2, \mu_3}$. Vi har tidligere sett at
	      $H=\cb{\rho_0, \rho_1, \rho_2}$ er en undergruppe. Vi har at $\rho_i H=H=H\rho_i$ siden
	      $\rho_i\in H$.

	      Vis at $\mu_i H=H\mu_i$.

	      Altså har vi at $H\leq G$ er en normal undergruppe.

	\item La oss fortsatt se på $S_3$, men sett nå $H=\cb{\rho_0, \mu_1}$. Da har vi at
	      $\rho_1 H=\cb{\rho_1, \mu_3}$, men at $H\rho_1 =\cb{\rho_1, \mu_2}$. Altså er ikke $H$
	      normal i dette tilfellet.
\end{enumerate}

\begin{theorem*}{14.12}{}
	Følgende er ekvivalent for $H\leq G$:
	\begin{enumerate}
		\item $H$ er normal
		\item $gHg^{-1}\ \forall g\in G$, hvor $gHg^{-1}=\cb{ghg^{-1}\mid h\in H}$
		\item $ghg^{-1}\in H\ \forall h\in H, g\in G$
	\end{enumerate}
\end{theorem*}

Fra tidligere så har vi:
\begin{enumerate}
	\item $g\in G$
	\item $gH=H\iff g\in H$
	\item $g_1H\cap g_2H\neq\emptyset\iff g_1H=g_2H\iff g_1^{-1}g_2\in H\iff g_2^{-1}g_1\in H\iff g_2\in g_1 H$
\end{enumerate}

\textbf{Eksempel}: La $G=S_3$. Da ser vi at
\begin{enumerate}
	\item $H=\cb{\rho_0, \rho_1, \rho_2}\implies \rho_1H=H$ og $\mu_i H=\cb{\mu_1, \mu_2, \mu_3}$
	\item La $H=\cb{\rho_1, \mu_1}$. Da ser vi at $\rho_1 H=\cb{\rho_1, \mu_3}$, altså at
	      $\mu_3\in \rho_1 H$, som må bety at $\rho_1 H=\mu_1 H$ (siden de enten er helt disjunkte
	      eller helt like).
\end{enumerate}

\begin{theorem*}{14.14 og korollar 14.5}{}
	Anta at $H\leq G$ er normal og la $G/H$ være mengden av restklasser til $H$ i $G$. For
	$g_1H$ og $g_2H$ i $G/H$, definer
	\begin{align}
		(g_1H)(g_2H)=(g_1g_2)H\in G/H
	\end{align}
	Dette er en binæroperasjon på $G/H$, som blir en gruppe sammen med denne.
\end{theorem*}

\textbf{Bevis}: Må vise at binæroperasjonen er veldefinert i følgende forstand: La
$g_1'\in g_1H$. Da vet vi at $g_1H=g_1'H$. Tilsvarende har vi at for $g_2'\in g_2H$ er
$g_2H=g_2'H$. Må vise
\begin{align}
	(g_1g_2)H=(g_1'g_2')H
\end{align}
Med andre ord så må vi vise at dersom to elementer i domenet er like så må de også ende opp
på samme sted i ko-domenet.

Siden $(g_1H)(g_2H)=(g_1'H)(g_2'H)$, så kan vi ta et element $g_1'g_2'h\in (g_1'g_2')H$. Siden
$g_1'\in g_1H$ og $g_2'\in g_2H$, så finnes $h_1, h_2\in H$ slik at $g_1'=g_1h_1$ og
$g_2'=g_2h_2$. Da har vi at
\begin{align}
	g_1'g_2'h & = (g_1h_1)(g_2h_2)h          \\
	          & = g_1(e)h_1g_2h_2h           \\
	          & = g_1(g_2g_2^{-1})h_1g_2h_2h \\
	          & = g_1g_2(g_2^{-1}h_1g_2)h_2h
\end{align}
Husk at siden $H$ er normal og $h_1\in H$, så må $g_2^{-1}h_1g_2\in H$. Dermed kan vi skrive
$\widetilde h := g_2^{-1}h_1g_2$. Videre har vi at $\widetilde{h}h_2h\in H$ siden $H$ er lukket.
Dermed får vi altså
\begin{align}
	g_1'g_2'h & = g_1g_2\widetilde{h}h_2h \in (g_1g_2)H
\end{align}
Derfor har vi at $(g_1'g_2')H\subseteq (g_1g_2)H$. Vi kan gjøre tilsvarende argument andre vei og
få $(g_1g_2)H\subseteq (g_1'g_2')H$, som tilsammen må bety at $(g_1g_2)H=(g_1'g_2')H$, som var
det vi ville vise. Dermed har vi vist at operatoren over er veldefinert.

La oss nå vise at dette blir en gruppe.
\begin{enumerate}[label=$\mathscr{G}$\arabic*)]
	\item Assosiativitet:
	      \begin{align}
		      g_1H\sb{(g_2H)(g_3H)} & = g_1H((g_2g_3)H)       \\
		                            & = g_1(g_2g_3)H          \\
		                            & = (g_1g_2)g_3H          \\
		                            & = \dots                 \\
		                            & = \sb{(g_1H)(g_2H)}g_3H
	      \end{align}
	\item Identitet: Vi har at $eH=H$ er identitetselementet:
	      \begin{align}
		      (eH)(gH)=(eg)H=gH=(ge)H=(gH)(eH)\ \forall g\in G
	      \end{align}
	\item Invers: Vi har at $(gH)^{-1}=g^{-1}H$:
	      \begin{align}
		      (gH)(g^{-1}H)=(gg^{-1})H=eH=(g^{-1}g)H=(g^{-1}H)(gH)
	      \end{align}
\end{enumerate}

\begin{definition}{Faktorgruppe/Kvotientgruppe}{}
	Dersom $G$ er en gruppe og $H$ er en normal undergruppe så kaller vi $G/H$ en
	\textbf{faktorgruppe} eller \textbf{kvotientgruppe}.
\end{definition}

\textbf{Merk}:
\begin{enumerate}
	\item Vi brukte at $H\leq G$ er normal for å få at binæroperasjonen er veldefinert.
	\item binæreoperatoren for faktorgrupper er rett og slett a gange sammen restklasser:
	      \begin{align}
		      (g_1H)(g_2H) & = \cb{g_1h_1g_2h_2\mid h_1, h_2\in H}            \\
		                   & = \cb{g_1g_2g_2^{-1}h_1g_2h_2\mid h_1, h_2\in H} \\
		                   & = \cb{g_1g_2h\mid h\in H}                        \\
		                   & = (g_1g_2)H
	      \end{align}
\end{enumerate}

\textbf{Eksempler}:
\begin{enumerate}
	\item La $G=\mathbb{Z}$ og see på $H=4 \mathbb{Z}$. Vi har at $H$ her er normal siden
	      $\mathbb{Z}$ er en abelsk gruppe. Den har fire restklasser:
	      \begin{align}
		       & 0+4 \mathbb{Z} = 4 \mathbb{Z} \\
		       & 1+4 \mathbb{Z}                \\
		       & 2+4 \mathbb{Z}                \\
		       & 3+4 \mathbb{Z}
	      \end{align}
	      som også er de fire elementene i $\mathbb{Z}/4 \mathbb{Z}$. La oss se på addisjon i
	      $\mathbb{Z}/4 \mathbb{Z}$:
	      \begin{align}
		      (1+4 \mathbb{Z}) + (2+4 \mathbb{Z}) & = (1+2)+4 \mathbb{Z} = 3+4 \mathbb{Z}                  \\
		      (0+4 \mathbb{Z}) + (2+4 \mathbb{Z}) & = (0+2)+4 \mathbb{Z} = 2+4 \mathbb{Z}                  \\
		      (2+4 \mathbb{Z}) + (2+4 \mathbb{Z}) & = (2+2)+4 \mathbb{Z} = 0+4 \mathbb{Z} = 4 \mathbb{Z}   \\
		      (3+4 \mathbb{Z}) + (2+4 \mathbb{Z}) & = (3+2)+4 \mathbb{Z} = 5+4 \mathbb{Z} = 1+4 \mathbb{Z} \\
	      \end{align}
	      Merk her at $-(2+4 \mathbb{Z}) = 2 + 4 \mathbb{Z}$, altså at dette er inversen siden
	      $4 \mathbb{Z}$ er identiteten. Vi ser at $\abs{\mathbb{Z}/4 \mathbb{Z}}=4$ og at dette
	      ligner på $(\mathbb{Z}_4,+_4)$!
	\item La nå $G=S_3$ hvor vi ser på $H=\cb{\rho_0, \rho_1, \rho_2}$. Vi har allerede sett
	      at denne er normal. Det finnes to elementer i $S_3/H$:
	      \begin{itemize}
		      \item $H=\rho_0 H=\cb{\rho_0, \rho_1, \rho_2}=\rho_1 H=\rho_2 H$
		      \item $\mu_1 H = \cb{\mu_1, \mu_2, \mu_3}=\mu_2 H=\mu_3 H$
	      \end{itemize}
	      Altså har vi at $\abs{S_3/H}=2$ med $\rho_0 H$ som identitetselement og
	      $(\mu_1 H)(\mu_2 H)= \mu_1^2 H = \rho_0 H$. Dette ligner på $(\mathbb{Z}_2, +_2)$!
\end{enumerate}

\textbf{Merk}:
\begin{enumerate}
	\item Dersom $\phi:G\ra G'$ er en homomorfi så er $\ker\phi$ normal i $G$.
	\item $\phi[G]:= \cb{\phi(g)\mid g\in G}\leq G'$. Vi har også trivielt at $\phi:G\ra \phi[G]$
	      er en surjektiv homomorfi.
\end{enumerate}

Fra dette får vi at dersom $H\leq G$ er normal så kan vi lage følgende homomorfi:
\begin{align}
	\Pi : G & \ra G/H    \\
	g       & \mapsto gH
\end{align}
Da ser vi at $\Pi(g_1g_2)=g_1g_2H=(g_1H)(g_2H)=\Pi(g_1)\Pi(g_2)$. Videre har vi at
\begin{align}
	\ker\Pi = \cb{g\in G\mid \Pi(g)=eH}=\cb{g\in G\mid g\in H} = H
\end{align}

\begin{theorem*}{14.11 - Fundamentalteoremet for (gruppe-)homomorfier}{}
	La $\phi: G\ra G'$ være en homomorfi og $H=\ker\phi$. Da er funksjonen
	\begin{align}
		\bar\phi : G/H & \ra \phi[G]     \\
		gH             & \mapsto \phi(g)
	\end{align}
	en veldefinert homomorfi og en isomorfi. Videre har vi at $\phi = \bar\phi \circ \Pi$.
\end{theorem*}

\textbf{Bevis}:
\begin{itemize}
	\item Veldefinert:
	      \begin{align}
		      g_1H=g_2H & \implies g_2\in g_1H                                          \\
		                & \implies g_2=g_1h                                             \\
		                & \implies \bar\phi(g_2H)=\phi(g_2)=\phi(g_1h)=\phi(g_1)\phi(h)
		      =\phi(g_1)=\bar\phi(g_1)
	      \end{align}
	\item Homomorfi:
	      \begin{align}
		      \bar\phi((g_1H)(g_2H)) & = \bar\phi((g_1g_2)H)          \\
		                             & = \phi(g_1g_2)                 \\
		                             & = \phi(g_1)\phi(g_2)           \\
		                             & = \bar\phi(g_1H)\bar\phi(g_2H)
	      \end{align}
	\item Bijektiv: Vis
	\item Kommutativt diagram:
	      \begin{align}
		      (\bar\phi \circ \Pi)(g) & = \bar\phi(\Pi(g)) \\
		                              & = \bar\phi(gH)     \\
		                              & = \phi(g)
	      \end{align}
	      som var det vi ville vise. \qed
\end{itemize}

\textbf{Eksempler}:
\begin{enumerate}
	\item Se på $\phi: \mathbb{Z}\ra \mathbb{Z}_4$ med $n\mapsto n\pmod{4}$. Dette er en homomorfi.
	      $\phi$ er surjektiv og $\ker\phi=\cb{n\in \mathbb{Z}\mid n \equiv 0 \pmod{4}}=4 \mathbb{Z}$.
	      Da følger det fra teoremet over at $\mathbb{Z}/4 \mathbb{Z} \ra \mathbb{Z}_4$ er en isomorfi
	      ved $\bar\phi$.
	\item Se på $\phi: \text{GL}(n, \mathbb{R})\ra \mathbb{R}^*$ ved $\phi(M)=\det M$. Dette er en
	      homomorfi som er surjektiv med $\ker\phi=\text{SL}(n, \mathbb{R})$. Så
	      $\bar\phi: \text{GL}(n, \mathbb{R})/\text{SL}(n, \mathbb{R})\ra \mathbb{R}^*$ er en
	      isomorfi.
	\item Se på
	      \begin{align}
		      \phi:S_n & \ra \mathbb{Z}_2                                                                \\
		      \sigma   & \mapsto \begin{cases}0 & \sigma\text{ like}\\1 & \sigma \text{ odde}\end{cases}
	      \end{align}
	      Da er $\phi$ en surjektiv homomorfi og $\ker\phi=A_n$. Så $S_n/A_n\ra \mathbb{Z}_2$ er
	      en isomorfi.
	\item Vi ser at $G/\cb{e}\cong G$ med $\phi(g)=g$ og $G/G\cong \cb{e}$ med $\phi(g)=e$.
\end{enumerate}

\textbf{Strategi for å vise at $G/H$ er isomorf med en gruppe $G'$}:
\begin{enumerate}
	\item Finn en surjektiv homomorfi $\phi: G\ra G'$ med $\ker\phi=H$
	\item Da gir funtamentalteoremet oss at $G/H\ra G'$ er en isomorfi siden $\phi[G]=G'$.
\end{enumerate}

\textbf{Eksempel - Eksamen Sommer 2023, oppgave 4}:
La $G$ være en endelig gruppe og $H_1, H_2\leq G$ normale undergrupper.
\begin{enumerate}[label=\alph*)]
	\item La $\phi: G\ra G/H_1\times G/H_2$ med $\phi(g)=(gH_1,gH_2)$. Vis at $\phi$ er en homomorfi.

	      Løsning:
	      \begin{align}
		      \phi(g_1g_2) & = (g_1g_2H_1, g_1g_2H_2)              \\
		                   & = ((g_1H_1)(g_2H_1),(g_1H_2)(g_2H_2)) \\
		                   & = (g_1H_1,g_1H_2)(g_2H_1,g_2H_2)      \\
		                   & = \phi(g_1)\phi(g_2)
	      \end{align}
	      Så $\phi$ er en homomorfi.
	\item Finn en injektiv homomorfi
	      \begin{align}
		      G/(H_1\cap H_2)\ra G/H_1\times G/H_2
	      \end{align}
	      og vis at denne er en isomorfi hvis og bare hvis
	      \begin{align}
		      \frac{\abs{H_1}\abs{H_2}}{\abs{H_1\cap H_2}} = \abs{G}
	      \end{align}

	      Løsning:
        \begin{align}
          \ker\phi &= \cb{g\in G\mid \phi(g) = e} \\
                   &= \cb{g\in G\mid \phi(g)=(H_1, H_2)} \\
                   &= \cb{g\in G\mid g\in H_1 \wedge g\in H_2} \\
                   &= \cb{g\in G\mid g\in H_1\cap H_2} \\
                   &= H_1\cap H_2
        \end{align}
        Fra fundamentalteoremet har vi da at $\bar\phi: G/H_1\cap H_2 \ra \phi[G]$ er en
        isomorfi. Men merk at $\phi[G]\subseteq G/H_1 \times G/H_2$, så vi kan ikke garantere
        at $\bar\phi: G/H_1\cap H_2 \ra G/H_1\times G/H_2$ er surjektiv. Den må likevel
        være injektiv, siden isomorfien er det. Vi har at
        \begin{align}
          \phi \text{ surjektiv} &\iff \abs{G/H_1\cap H_2} = \abs{G/H_1\times G/H_2} \\
                                 &\iff \frac{\abs{G}}{\abs{H_1\cap H_2}} 
                                 =\frac{\abs{G}}{\abs{H_1}}\cdot \frac{\abs{G}}{\abs{H_2}} \\
                                 &\iff \frac{\abs{H_1}\abs{H_2}}{\abs{H_1\cap H_2}}
                                 = \frac{\abs{G}^2}{\abs{G}} = \abs{G} \qed
        \end{align}
\end{enumerate}

\begin{definition}{Simpel}{}
  Vi sier at en gruppe $G$ er \textbf{simpel} dersom det ikke finnes en normal undergruppe $H$ 
  slik at 
  \begin{align}
    \cb{e}<H<G
  \end{align}
\end{definition}

\textbf{Eksempel}: Dersom $\abs{G}=p$ hvor $p$ er et primtall så vet vi fra lagrange at 
$G$ må være simpel, siden det ikke kan finnes noen undergrupper $H$ med $\cb{e}<H<G$. 

\begin{theorem*}{15.15}{}
 $A_n$ er simpel når $n\geq 5$. 
\end{theorem*}

\begin{theorem*}{Klassifisering av endelige simple grupper}{}
 La $G$ være en endelig, simpel gruppe. Da er den isomorf med en av følgende:
 \begin{enumerate}
   \item $\mathbb{Z}_p$ hvor $p$ er et primtall
   \item $A_n$ når $n\geq 5$
   \item En simpel gruppe av Lie-type
   \item En av de 26 sporadiske gruppene
 \end{enumerate}
\end{theorem*}
