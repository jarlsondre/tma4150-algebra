\section{Ringer og Kropper}

\begin{definition}{Ring}{}
  En \textbf{ring} er en ikke-tom mengde $R$ med to operasjoner, + og $\cdot$, slik at følgende
  holder:
  \begin{enumerate}[label=$\mathscr{R}$\arabic*)]
    \item $(R, +)$ er en abelsk gruppe
    \item Den andre operatoren, $\cdot$, skal være assossiativ, altså at 
      $(a\cdot b)\cdot c = a\cdot (b\cdot c)$ for alle $a,b,c\in R$
    \item De følgende distributive lovene skal holde:
      \begin{itemize}
        \item $a\cdot (b+c) = a\cdot b + a\cdot c$
        \item $(a + b)\cdot c = a\cdot c + b\cdot c$
      \end{itemize}
  \end{enumerate}
  Dersom vi også har at $a\cdot b = b \cdot a$ for alle $a, b\in R$, så sier vi at $R$ er en
  \textbf{kommutativ ring}.
\end{definition}

\textbf{Eksempler}:
\begin{enumerate}
  \item $\mathbb{Z}, \mathbb{Q}, \mathbb{R}, \mathbb{C}$ med vanlig addisjon og multiplikasjon
    er alle kommutative ringer.
  \item $M_n(\mathbb{R})$, altså alle $n\times n$ matriser over $\mathbb{R}$, er en ring, men 
    den er ikke kommutativ.
\end{enumerate}

\textbf{Merk}:
\begin{enumerate}
  \item Vanligvis så kaller vi + "addisjon" og $\cdot$ "multiplikasjon". Vi skriver også 
    $ab$ for $a\cdot b$.
  \item Siden $(R, +)$ skal være en abelsk gruppe, så må det finnes en identitet for denne
    operatoren. Denne kaller vi vanligvis for 0. 
  \item Ringene vi ser på i dette faget vil også ha identiteter for den multiplikative operatoren
    som vi kaller 1, slik at $1\cdot a = a\cdot 1 = a$ for alle $a \in R$. 
\end{enumerate}

\textbf{Eksempler}:
\begin{enumerate}
  \item Den multiplikative identiteten i $M_2(\mathbb{R})$ er 
    $I = \begin{pmatrix} 1 & 0 \\ 0 & 1 \end{pmatrix}$.
  \item La $n \geq 2$ og $\mathbb{Z}_n = \cb{0, 1, \dots, n-1}$. Vi har tidligere sett at
    $(\mathbb{Z}_n, +_n)$ er en abelsk gruppe. La $\cdot_n$ være multiplikasjon modulo $n$. Da
    har vi at $(\mathbb{Z}_n, +_n, \cdot_n)$ er en kommutativ ring.
\end{enumerate}

\begin{definition}{Enhet}{}
  La $R$ være en ring. Et element $a\in R$ kalles en \textbf{enhet} dersom det finnes $b\in R$
  slik at $ab = ba = 1$.
\end{definition}

\begin{definition}{Divisjonsring}{}
  La $R$ være en ring. Da sier vi at $R$ er en \textbf{divisjonsring} dersom alle elementene i
  $R\setminus \cb{0}$ er enheter.
\end{definition}

\begin{definition}{Kropp}{}
  La $R$ være en ring. Da sier vi at $R$ er en \textbf{kropp} dersom den er en kommutativ
  divisjonsring. 
\end{definition}

\textbf{Eksempler}
\begin{enumerate}
  \item $\mathbb{Z}$ er ikke en kropp siden det kun er 1 og -1 som er enheter.
  \item $\mathbb{Q}, \mathbb{R}, \mathbb{C}$ er kropper.
  \item $M_n(\mathbb{R})$ er ikke en divisjonsring og dermed heller ikke en kropp. 
\end{enumerate}

\textbf{Vis}
\begin{enumerate}
  \item $R$ er en kropp $\iff (R,+)$ og $(R \setminus \cb{0},\cdot)$ er abelske grupper og
    $a(b+c) = ab + ac$. 
  \item La $U(R) = \cb{a\in R\mid a \text{ er en enhet}}$. Da er $(U(R), \cdot)$ en gruppe, men 
    ikke nødvendigvis abelsk. 
  \item Dersom $a\in U(R)$ så finnes det kun én $b\in R$ med $ab = 1 = ba$. 
\end{enumerate}

\begin{theorem*}{18.8}{}
  La $R$ være en gruppe og $a, b \in R$. Da holder følgende:
  \begin{enumerate}
    \item $0\cdot a = a\cdot 0 = 0$
    \item $a\cdot (-b) = (-a)\cdot b = -a\cdot b$
    \item $(-a)\cdot(-b) = ab$
  \end{enumerate}
\end{theorem*}

\begin{definition}{Ringhomomorfi}{}
  La $R$ og $S$ være ringer. Da sier vi at en funksjon $\phi: R \ra S$ er en 
  \textbf{ringhomomorfi} dersom:
  \begin{enumerate}
    \item $\phi(a+b) = \phi(a)+\phi(b)$
    \item $\phi(ab)=\phi(a)\phi(b)$
  \end{enumerate}
  for alle $a, b \in R$. Dersom $\phi$ også er bijektiv så sier vi at det er en isomorfi.
  Kjernen av $\phi$ er alle elementene som sendes til 0.
\end{definition}

\textbf{Eksempler}:
\begin{enumerate}
  \item Har sett at $\phi: \mathbb{Z}\ra \mathbb{Z}_n, a \mapsto a\pmod{n}$ er en gruppehomomorfi.
    Vis at $\phi(ab) = \phi(a) \cdot_n \phi(b)\ \forall a, b\in \mathbb{Z}$. Kjernen til $\phi$ er
    $n \mathbb{Z}$. 
  \item La $R = \cb{\begin{pmatrix}a & b \\ 0 & c \end{pmatrix}\mid a, b, c\in \mathbb{R}}$. Da
    er dette en ring. Definer nå $\phi: R \ra \mathbb{R}$ ved 
    $\begin{pmatrix} a & b \\ 0 & c \end{pmatrix}\mapsto a$. Da ser vi (ved litt regning) at 
    $\phi(A + B) = \phi(A) + \phi(B)$ og at $\phi(AB) = \phi(A)\phi(B)$. Altså er $\phi$ en
    ringhomomorfi.
  \item La $\phi : \mathbb{C} \ra \mathbb{C}$ ved $z \mapsto \overline{z}$, altså konjugasjon.
    Da er dette en bijektiv ringhomomorfi, altså en isomorfi. 
\end{enumerate}

