\section{Gruppevirkninger}
\begin{definition}{Gruppevirkning}{}
	La $G$ være en gruppe og $X$ være en mengde. En \textbf{gruppevirkning} på $X$ fra
	$G$ er en funksjon $G\times X \ra X$ som tilfredsstiller to krav:
	\begin{itemize}
		\item $(e, x) = x$ for alle $x \in X$.
		\item $(g_1g_2, x) = (g_1, (g_2, x))$ for alle $g_1, g_2 \in G, x\in X$.
	\end{itemize}
	Dersom disse kravene tilfredsstilles så kaller vi $X$ en \textbf{$G$-mengde}.
	\newline\newline
	\textbf{Notasjon}: Dersom $g\in G$ og $x\in X$ så skriver vi normalt sett
	$(g, x) = gx$.
\end{definition}

\textbf{Eksempler}:
\begin{enumerate}
	\item La $G = S_n$ og $X = \cb{1, \dots, n}$. For en permutasjon $\sigma \in G$ og
	      $x\in X$ så er $\sigma x = \sigma(x)$ en gruppevirkning.
	\item La $G$ være en gruppe og $X=\cb{H\leq G}$ være mengden av alle undergrupper av
	      $G$. La oss videre definere en funksjon
	      \begin{align}
		      G\times X   & \ra X                                       \\
		      gH = (g, H) & \mapsto gHg^{-1} = \cb{ghg^{-1}\mid h\in H}
	      \end{align}
	      Sjekk at dette er en undergruppe av $G$. Vi har at $eH = eHe^{-1} = H$ og at
	      $(g_1g_2)H = g_1g_2Hg_2^{-1}g_1^{-1} = g_1(g_2Hg_2^{-1})g_1^{-1} = g_1(g_2H)$.
	      Denne gruppevirkningen kalles \textbf{Konjugasjon}.
	\item La
	      \begin{align}
		      G & = \cb{A\in \mathcal{M}_3(\mathbb{R})\mid A \text{ er ortogonal}} \\
		        & = \cb{A\in \mathcal{M}_3(\mathbb{R})\mid A^{-1} = A^\intercal}   \\
		        & = \text{O}_3(\mathbb{R})
	      \end{align}
	      være en gruppe med matrisemultiplikasjon som binæroperator. Merk at
	      $\norm{Av} = \norm{v} \forall v\in \mathbb{R}^3, A\in G$, altså at
	      $A$ bevarer normen til vektorer. Fiksér en $a > 0$ og la
	      $X = \cb{r\in \mathbb{R}^3\mid \norm{r} = a}$. For $A \in G$ og $v\in X$
	      så er $\norm{Av} = \norm{v}$, så $Av\in X$.

	      Videre har vi at:
	      \begin{enumerate}
		      \item $Iv = v \forall v\in X$
		      \item $(AB)v = A(Bv)\ \forall A, B\in G, v\in X$
	      \end{enumerate}
	      Dermed følger det at $X$ er en $G$-mengde og at vi har en gruppevirkning.
	\item La $G = \mathbb{Z}_2$ og $X = \mathbb{R}$. Definer så
	      \begin{align}
		      G\times X & \ra X                \\
		      mx        & \mapsto \begin{cases}
			                          x  & m = 0 \\
			                          -x & m = 1
		                          \end{cases}
	      \end{align}
	      Da er $0x = x$ og (sjekk) $(m + n)x = m(nx)$.
\end{enumerate}

\begin{definition}{Transitiv Virkning}{}
	La $G$ være en gruppe og $X$ en mengde. Vi sier at $G$ virker \textbf{transitivt} på
	$X$ dersom $\forall x_1, x_2\in X\ \exists g\in G$ hvor $gx_1 = x_2$.
\end{definition}

\textbf{Eksempel}: Vi har at (1) og (3) fra det forrige eksempelet er transitive virkninger.
Spesielt så ser vi at $forall v_1, v_2 \in \mathbb{R}^3$ med $\norm{v_1}=\norm{v_2}$ så vil det
finnes en $A \in \text{O}_3(\mathbb{R})$ slik at $Av_1 = v_2$ (ikke helt trivielt).

Videre har vi at (2) ikke er transitiv. Vi ser at $\abs{H} = \abs{gHg^{-1}}$, så derfor kan man
ikke gå fra en størrelse til en annen.

\begin{definition}{}{}
	La $G$ være en gruppe og $X$ en mengde, og la $g\in G$ og $x\in X$. Da definerer vi følgende
	mengder:
	\begin{align}
		G_x & = \cb{h\in G\mid hx = x} \subseteq G  \\
		X_g & = \cb{y \in X\mid gy = y} \subseteq X
	\end{align}
\end{definition}

\begin{theorem*}{}{}
	La $G$ være en gruppe, $X$ være en mengde og $x \in X$. Da vil $G_x \leq G$.
\end{theorem*}

\textbf{Bevis}:
La oss først merke at $G_x \neq \emptyset$ siden $e \in G_x$. La nå $h_1, h_2 \in G_x$. Da er
$h_i x = x$, noe som betyr at $x = h_i^{-1}x$. Da vil
\begin{align}
	(h_1h_2^{-1})x & = h_1(h_2^{-1}x) \\
	               & = h_1(x)         \\
	               & = x
\end{align}
noe som betyr at $h_1h_2^{-1} \in G_x$. Altså må $G_x \leq G$. \qed

\begin{definition}{Isotropi-undergruppen}{}
	$G_x$ kalles \textbf{isotropi-undergruppen} til $x$ i $G$.
\end{definition}

\begin{definition}{Bane til element in mengde}{}
	La $G$ være en gruppe, $X$ være en mengde og $x \in X$. Da sier vi at \textbf{banen} til $x$
	er $Gx = \cb{gx\mid g\in G}$.
\end{definition}

\textbf{Eksempler}:
\begin{enumerate}
	\item La $G$ være en gruppe og definer $X = \cb{H \leq G}$. Da vet vi at $g\cdot H = gHg^{-1}$
	      for $H\in X$. Da er $G_H = \cb{g\in G\mid gHg^{-1} = H}$.
	\item La $G = \text{O}_3(\mathbb{R})$ og $X = \cb{v\in \mathbb{R}^3\mid \norm{v} = a}$. Da har
	      vi at
	      \begin{align}
		      G_v & = \cb{A\in \text{O}_3(\mathbb{R})\mid Av = v} \\
		          & = \cb{A\in\text{O}_3(\mathbb{R})\
			      |\ v \text{ egenvektor av }A\text{ med }\lambda=1}
	      \end{align}
	\item Vi har at 'transitiv virkning $\iff Gx=X\ \forall x\in X$'
\end{enumerate}

\begin{theorem*}{16.16}{}
	La $G$ være en gruppe, $X$ være en mengde og $x\in X$ et element. Da har vi at
	$\abs{Gx} = (G:G_x)$, hvor $(G:G_x)$ betegner indeksen til $G_x$ i $G$, altså antall venstre
	restklasser.
\end{theorem*}
\textbf{Bevis}:
For $g_1, g_2\in G$ så har vi at 
$g_1x = g_2x \iff g_2^{-1}g_1x = x \iff g_2^{-1}g_1\in G_x \iff g_1G_x = g_2G_x$. Dermed har vi
altså en veldefinert funksjon:
\begin{align}
  Gx &\ra \cb{\text{venstre restklasser til }G_x\text{ i }G} \\
  gx &\mapsto gG_x
\end{align}
Ettersom denne funksjonen er bijektiv så må mengdene være like store, som var det vi ville 
vise. \qed

Vis følgende:
\begin{enumerate}
  \item $\forall x_1, x_2 \in X, x_1 \in Gx_2 \iff Gx_1 = Gx_2 \iff x_2 \in Gx_1$
  \item Definer relasjonen på $X$ ved $x_1 \in Gx_2$. Dette er en ekvivalensrelasjon.
  \item De følgende utsagnene er ekvivalente:
    \begin{enumerate}
      \item $G$ virker transitivt på $X$
      \item $Gx = X\ \forall x\in X$
      \item Det finnes $x\in X$ med $Gx = X$
      \item $X$ har kun én ekvivalensklasse for relasjonen i 2)
    \end{enumerate}
\end{enumerate}

\subsection{Burnsides Formel}

\begin{theorem*}{Burnsides Formel}{}
  La $G$ være en endelig gruppe og $X$ en endelig $G$-mengde med $r$ baner i $X$. Da følger det at:
  \begin{align}
    r\cdot \abs{G} = \sum_{g\in G}\abs{X_g}
  \end{align}
\end{theorem*}

\textbf{Bevis}: Se på undermengden $M$ av $G\times X$:
\begin{align}
  M = \cb{(g,x)\mid gx=x}
\end{align}
Da har vi at 
\begin{align}
  \sum_{g\in G}\abs{X_g} &= \sum_{g\in G}\abs{\cb{x\in X\mid gx=x}} = \abs{M} \\
                         &= \sum_{x\in X}\abs{\cb{g\in G\mid gx=x}} \\
                         &= \sum_{x_in X}\abs{G_x}
\end{align}
Videre har vi at for alle $x\in X$ så er $\abs{Gx} = (G:G_x) = \frac{\abs{G}}{\abs{G_x}}$. Da
må vi ha at 
\begin{align}
  \abs{M} &= \sum_{x\in X}\abs{G_x} = \sum_{x\in X}\frac{\abs{G}}{\abs{Gx}} 
          = \abs{G}\cdot \sum_{x\in X} \frac{1}{\abs{Gx}}
\end{align}
La nå $B$ være en av banene i $X$, altså at $B = Gx$ for en $x \in X$. Da har vi at alle elementene
i $B$ har $B$ selv som bane, altså at $B = Gx'\ \forall x'\in B$ og at
\begin{align}
  \sum_{x\in B}\frac{1}{\abs{Gx}} = \sum_{x\in B}\frac1B = 1
\end{align}
Dette betyr altså at hver bane, $B$, gir et bidrag på 1 i $\sum_{x\in X} \frac{1}{\abs{Gx}}$. Siden
banene må være helt disjunkte og vi har $r$ baner så betyr dette at 
\begin{align}
  \abs{M} = \abs{G}\sum_{x\in X}\frac{1}{\abs{Gx}} = \abs{G}\cdot r
\end{align}
og siden $\abs{M} = \sum_{g\in G}\abs{X_g}$ så må altså
\begin{align}
  \sum_{g\in G}\abs{X_g} = \abs{G}\cdot r
\end{align}
som var det vi ville vise. \qed

\textbf{Eksempel (Eksamen Vår 2013, oppg. 4)}:

Anta et perlekjede skal være bestående av 11 like store perler, hvorav 5 skal være sorte og 6 skal
være hvite. Hvor mange ulike slike perlekjeder kan du lage?

Først, sett $G$ til å være symmetrigruppen til perlekjedet. Da vil elementene i $G$ bestå av
11 rotasjoner, $\cb{\rho_0, \dots, \rho_10}$ og 11 speilinger, $\cb{\mu_1, \dots, \mu_11}$. Her
tenker vi at $\mu_i$ holder perle $i$ i ro. Da vil $\abs{G} = 22$. 

La nå $X$ være mengden av alle fargelagte perlekjeder uten å ta hensyn til symmetrier. Da vil $G$
virke på $X$ og antall baner vil være antall ulike perlekjeder som vi skal frem til. Fra Burnsides
formel har vi 
\begin{align}
  22\cdot r = \sum_{g\in G} \abs{X_g} = \sum_{i=0}^{10} \abs{X_{\rho_i}} + 
  \sum_{j=1}^{11}\abs{X_{\mu_j}}
\end{align}
Merk at $\abs{X} = \binom{11}{5}$ og at
\begin{align}
  \abs{X_{\rho_i}} = \begin{cases}
      \abs{X} = \binom{11}{5} & i = 0 \\
      0 & i \neq 0
  \end{cases}
\end{align}
siden ingen elementer holdes i ro da det finnes 5 sorte og 6 hvite. 

Hva med $X_{\mu_i}$? Dersom $x\in X$ skal ligge i $X_{\mu_i}$ så må perle $i$ være sort slik at vi
kan ha et partall antall sorte på hver side, også må det faktisk være et partall på hver side,
altså to perler i dette tilfellet. Det vil finnes $\binom{5}{2}$ slike perlekjeder, fordi man
vil ha fem perler på hver side og man har kun frihet til å velge den ene siden, siden den andre
må være lik. Når man velger den ene siden, som består av fem perler, så kan man velge hvor de to
sorte skal være. Altså har vi at
\begin{align}
  \abs{X_{\mu_i}} = \binom{5}{2}
\end{align}
Dersom vi nå setter alt inn i Burnsides formel får vi:
\begin{align}
  22r &= \binom{11}{5} + \sum_{i=1}^{11} \binom{5}{2} \\
      &= 572
\end{align}
som betyr at $r = 26$. Altså finnes det 26 slike perlekjeder. 

