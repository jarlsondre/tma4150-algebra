\section{Sykliske Grupper}

\textbf{Notasjon}: Dersom $G$ er en gruppe, $a \in G$ og vi bruker multiplikativ
notasjon, så skriver vi $\inner{a} = \cb{a^t\mid t\in \mathbb{Z}}$.

\begin{theorem*}{5.17}{}
	La $G$ være en gruppe og $a \in G$. Da er $\inner{a}$ en undergruppe av $G$.
\end{theorem*}
\textbf{Bevis}: Vi har at $\inner{a} \neq \emptyset$ og for $a^m, a^n \in \inner{a}$
så er $a^m(a^n)^{-1} = a^ma^{-n} = a^{m-n} \in \inner{a}$. Dermed har vi fra Teorem
5.14 at $\inner{a} \leq G$. \qed

\begin{definition}{Syklisk Undergruppe}{}
	Vi kaller $\inner{a}$ den \textbf{sykliske undergruppen} av $G$ generert av $a$, og
	vi sier at $G$ er \textbf{syklisk} dersom det finnes en $a \in G$ slik at
	$G = \inner{a}$.
\end{definition}

\textbf{Merk}:
\begin{enumerate}
	\item En syklisk gruppe kan ha flere generatorer
	\item $\abs{\inner{a}}$ kan være endelig, f.eks. at $a^s = a^t$ med $s \neq t$.
\end{enumerate}

\textbf{Eksempler}:
\begin{enumerate}
	\item 1 og -1 er generatorer i $\mathbb{Z}$, så $\mathbb{Z}$ er syklisk.
	\item For $\mathbb{Z}_9$ så har vi at
	      \begin{itemize}
		      \item $\inner{3} = \cb{0, 3, 6} = \inner{6}$
		      \item $\inner{0} = \cb{0}$
		      \item $\inner{1} = \inner{2} = \inner{4} = \inner{5} = \inner{7} = \inner{8}$
	      \end{itemize}
	      Vi ser at $\inner{a} = \mathbb{Z}_9 \iff \gcd(a, 9) = 1$
	\item På øving: $a \in \mathbb{Z}_n$ er en generator $\iff \gcd (n, 1) = 1$
	\item La $U=\cb{z\in\mathbb{C}\mid \abs{z} = 1}$. Da er
	      $\inner{i} = \cb{i, 1, -1, -i} = U_4$.
\end{enumerate}

\begin{definition}{Ordenen til et gruppeelement}{}
	La $G$ være en gruppe og $a \in G$. Da sier vi at \textbf{ordenen} til $a$ er
	$\abs{\inner{a}}$.
\end{definition}

\textbf{Eksempel}: Vi har sett på $\mathbb{Z}_9$. Der så vi at
$\inner{3} = \inner{6} = \cb{0, 3, 6}$. Dermed har vi at 3 og 6 har orden 3.

\begin{theorem*}{6.1}{}
	La $G$ være en gruppe. Hvis $G$ er syklisk, så er $G$ abelsk.
\end{theorem*}

\textbf{Bevis}: Hvis $G = \inner{a}$, altså syklisk, så kan vi skrive alle elementer
i $G$ som en potens av $a$. La $x, y \in G$. Da kan vi skrive $x = a^n$ og $y = a^m$.
Dermed har vi at $xy = a^na^m = a^{n+m} = a^{m+n} = a^ma^n = yx$. Altså må $G$ være
abelsk. \qed


\begin{theorem*}{}{}
	La $G$ være en gruppe og $H \leq G$ en undergruppe. Da er $H$ syklisk.
\end{theorem*}

\textbf{Bevis}:
Dersom $H = \cb{e}$, så er $H$ trivielt syklisk, så la oss derfor anta at $H$ ikke
er triviell. La $a \in G$ med $G = \inner{a}$, altså at $a$ er en generator av $G$.
Siden $H \neq \cb{e}$, så vil det finnes $a^n \in H$ med $n \neq 0$. Siden $H$ er en
undergruppe og derfor også en gruppe, så må også $a^{-n} = (a^n)^{-1} \in H$. Med andre
ord så vil ikke mengden $I=\cb{n \in \mathbb{N}\mid a^n \in H}$ være tom, altså
$I \neq \emptyset$. Videre må det derfor finnes et minste element i $I$, $n_0$. Vi
skal se at $a^{n_0}$ kommer til å generere $H$.

Siden $a^{n_0} \in H$, så vil $\inner{a^{n_0}} \leq H$. Se nå på et tilfeldig
element $a^m \in H$. Fra divisjonsalgoritmen har vi at $m = qn_0 + r$ for
$0 \leq r \leq n_0-1$. Dette betyr også at $r = m - qn_0$. Merk at
$a^{qn_0} = (a^{n_0})^q \in H$ siden $a^{n_0}\in H$, så $a^{-qn_0}\in H$ også. I
tillegg har vi at $a^r=a^{m-qn_0}=a^ma^{-qn_0}\in H$.

Dersom $r > 0$ så kan ikke $a^r \in H$, fordi $r<n_0$ og vi har antatt at $n_0$ er
den minste. Så vi har at $r = 0$, men dette betyr at $m = qn_0$, som igjen betyr at
$a^{n_0}$ genererer $H$, som var det vi ville vise.\qed

\begin{theorem*}{(Korollar) 6.7}{}
	Alle undergrupper av $\mathbb{Z}$ er på formen
	$m \mathbb{Z} = \cb{\dots, -2m, -m, 0, m, 2m, \dots}$.
\end{theorem*}

\textbf{Vis følgende utsagn}:
\begin{enumerate}
	\item Dersom $p\in \mathbb{N}$ er et primtall så vil $\mathbb{Z}_p$ kun ha to
	      undergrupper, $\cb{0}$ og $\mathbb{Z}_p$.
	\item La $m, n \in \mathbb{Z}$ og definer $H = \cb{am + bn\mid a, b \in \mathbb{Z}}$.
	      Da vil $H \leq Z$ og det vil finnes $d \in \mathbb{Z}$ med $H = \inner{d}$. Vis
	      at $d = \gcd(m, n)$ er en gyldig kandidat.
	\item Vi har at $a\in \mathbb{Z}_n$ er en generator hvis og bare hvis $\gcd(a,n)=1$.
	      Derfor kan vi si at $\phi(n) = \text{ antall generatorer i }\mathbb{Z}_n$.
	      (Hint: $\gcd(a, b) = 1 \iff \exists r, s \in \mathbb{Z} : ra + sb = 1$)
\end{enumerate}

\begin{theorem*}{6.10}{}
	La $G$ være en gruppe. Dersom $G$ er syklisk så har vi følgende:
	\begin{itemize}
		\item $\abs{G} = \infty \implies G$ er isomorf med $\mathbb{Z}$.
		\item $\abs{G} = n \implies G$ er isomorf med $\mathbb{Z}_n$.
	\end{itemize}
\end{theorem*}

\textbf{Eksempel}:
Se på gruppen $U_4 = \cb{z\in\mathbb{C}\mid z^4=1} = \cb{\pm 1, \pm i}$ med
multiplikasjon som binæroperator. Siden denne er syklisk (med $\pm i$ som generator)
og $\abs{U_4} = 4$, så må $(U_4, \cdot) \cong (\mathbb{Z}_4, +_4)$. En slik isomorfi
er gitt ved
\begin{align}
	f: U_4 & \ra \mathbb{Z}_4 \\
	i^n    & \mapsto n
\end{align}
Vis at $f$ er en isomorfi ved å vise at den er bijektiv og homomorf.

\begin{theorem*}{6.14}{}
	La $G$ være en gruppe generert av $a$ med $\abs{G} = n$ og la $a^t \in G$ være et
	vilkårlig element. Da er ordenen til $a^t$ gitt ved
	\begin{align}
		\abs{\inner{a^t}} & = \frac{n}{\gcd{t, n}}
	\end{align}
	Videre så gjelder $\inner{a^t} = \inner{a^s} \iff \gcd(n, t) = \gcd(n, s)$.
\end{theorem*}

\begin{theorem*}{(Korollar) 6.16}{}
  La $G$ være en gruppe. Dersom $G$ er syklisk av orden $n$ og $a^t \in G$ så har vi
  at 
  \begin{align}
    \inner{a^t} = G \iff \gcd(n, t) = 1
  \end{align}
\end{theorem*}
